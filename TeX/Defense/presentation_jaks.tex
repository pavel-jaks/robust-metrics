\documentclass[czech]{beamer}

\usetheme{Rochester}

\usepackage{amsmath}
\usepackage{amsopn}
\usepackage{caption}
\usepackage{subcaption}
\usepackage{babel}

\title{Moderní metody robustního strojového učení}
\author{Bc. Pavel Jakš}
\institute{Matematická informatika, FJFI ČVUT v Praze}

\begin{document}

\selectlanguage{czech}

\frame{\titlepage}

\begin{frame}
    \frametitle{Obsah}
    \tableofcontents
\end{frame}

\section{Kontext}

\begin{frame}
    \frametitle{Klasifikace v prostředí neuronových sítí}
    \begin{itemize}
        \item Klasifikační neuronová síť: $F_\theta : X \rightarrow Y$
        \begin{itemize}
            \item $X$ - vzorky; vzory zobrazení neuronové sítě
            \item $Y$ - množina pravděpodobnostních rozdělení na třídách
        \end{itemize}
        \item Učení - optimalizace kritéria na trénovací datové sadě
        \begin{itemize}
            \item $\hat{\theta} = \operatornamewithlimits{argmin}_{\theta} J(\theta)$
            \item $J(\theta) = \frac{1}{N} \sum_{i=1}^{N} L(y^{(i)}, F_\theta(x^{(i)}))$
        \end{itemize}
        \item Klasifikace je potom zobrazení $C: Y \rightarrow \{1, 2, ..., m\}$
        \begin{itemize}
            \item $C(y) = \operatornamewithlimits{argmax}_{i \in \{1, ..., m\}} y_i$
        \end{itemize} 
    \end{itemize}
\end{frame}

\section{Adversariální vzorky}

\begin{frame}
    \frametitle{Adversariální vzorek}
    \begin{itemize}
        \item Szegedy a spol. objevili zvláštní chování klasifikačních sítí \cite{szegedy2014intriguing}
        \item Pojmenujme \emph{benigním} takový vzorek, který je správně klasifikovaný
        \item Mějme benigní vzorek $x$, pak \emph{adversariální} vzorek $\tilde{x}$ je takový, že
        $\rho(x, \tilde{x}) \leq \kappa$ a $C(F_\theta(x)) \neq C(F_\theta(\tilde{x}))$
    \end{itemize}
\end{frame}

% \subsection{Generování adversariálních vzorků}

\begin{frame}
    \frametitle{Jak získat adversariální vzorek?}
    \begin{itemize}
        \item Cílená optimalizační metoda
        \begin{itemize}
            \item $\tilde{x} = \operatornamewithlimits{argmin}_{\hat{x}} \rho(x, \hat{x}) + \lambda \cdot L(\tilde{y}, F_{\theta}(\hat{x}))$
        \end{itemize}
        \item Metoda CW
        \begin{itemize}
            \item $\tilde{x} = \operatorname{argmin}_{\hat{x}} \rho(x, \hat{x}) - \lambda \cdot L(y, F_{\theta}(\hat{x}))$
        \end{itemize}
    \end{itemize}
\end{frame}

\section{Metriky vizuální podobnosti}

\begin{frame}
    \frametitle{Použité metriky vizuální podobnosti}
    \begin{itemize}
        \item Metriky založené na $l_p$ normách
        \begin{itemize}
            \item $l_1$, $l_2$, $l_\infty$
        \end{itemize}
        \item Metriky založené na indexu \emph{SSIM}
        \begin{itemize}
            \item Jezdící okno variabilní velikosti porovnávající jas, kontrast a strukturu dvou obrázků
        \end{itemize}
        \item Metrika založená na $PSNR$
        \begin{itemize}
            \item Logaritmická transformace metriky založené na $l_2$ normě
        \end{itemize}
    \end{itemize}
\end{frame}

\begin{frame}
    \frametitle{Příklad vzorku $l_2$}
    \begin{figure}
        \centering
        \includegraphics[width=0.5\textwidth]{Images/adv_pic.png}
        \centering
        \caption{Vzorek vygenerovaný metodou CW za užití metriky založené na $l_2$}
    \end{figure}
\end{frame}

\begin{frame}
    \frametitle{Příklad vzorku SSIM}
    \begin{figure}
        \centering
        \includegraphics[width=0.5\textwidth]{Images/dssim.png}
        \centering
        \caption{Vzorek vygenerovaný metodou CW za užití metriky založené na indexu SSIM (velikost jezdícího okna 5)}
    \end{figure}
\end{frame}



\section*{Závěr}

\begin{frame}
    \frametitle{Závěr}
    \begin{itemize}
        \item Různé metriky vizuální podobnosti použité při generování adversariálních vzorků vedou na různé výsledky
    \end{itemize}
\end{frame}

\section*{Literatura}

\begin{frame}
    \frametitle{Literatura}
    \begin{thebibliography}{1}
        % \bibitem{Goodfellow}I. Goodfellow, Y. Bengio, A. Courville,
        % \emph{Deep Learning}. MIT Press, 2016.

        % \bibitem{MNIST} Y. Lecun, C. Cortes, C. J. Burges,
        % \emph{The mnist database of handwritten digits}. 1998.

        \bibitem{szegedy2014intriguing} C. Szegedy, W. Zaremba, I. Sutskever, J. Bruna, D. Erhan, I. Goodfellow, R. Fergus,
        \emph{Intriguing properties of neural networks}.
        arXiv, 2014.

        % \bibitem{GoodfellowEtAl} I. Goodfellow, J. Shlens, C. Szegedy,
        % \emph{Explaining and Harnessing Adversarial Examples}.
        % In 'International Conference on Learning Representations', ICLR 2015.
    \end{thebibliography}
\end{frame}

\section{Otázky}

\begin{frame}
    \frametitle{Automatická derivace duální Sinkhornovy metriky}
    \begin{itemize}
        \item Q: Proč nebyla využita možnost naimplementovat derivaci duální Sinkhornovy metriky v knihoně Pytorch?
        \item A: Technicky byla tato metrika sama o sobě automaticky derivovatelná, pouze vracela v gradientu hodnoty NaN.
        Kdybych pak naimplementoval vlastní derivaci, očekával bych stejné chování.
    \end{itemize}
\end{frame}

\begin{frame}
    \frametitle{Vliv metrik na výsledné adversariální vzorky}

    \begin{itemize}
        \item Q: Proč rozdílné optimalizační funkce generují jiné adversariální obrázky? Jaké vygenerované
        obrázky můžeme očekávat pro jakou metriku?
        \item A: Způsob implementace řešení optimalizačního problému generování adversariálních vzorků
        spoléhal na gradientní sestup, resp. na jeho znaménkovou variantu. Různé metriky mají potom různé gradienty,
        což je podstata různorodosti v obrázcích.
        Často se stávalo (např. pro SSIM 5), že derivace metriky byla na některých místech nulová,
        proto se neblížil daný obrázek v daném místě k původnímu, což vedlo ke znatelným artefaktům v obrázku.
        Čili, lze očekávat, že metrika, která má derivaci řádově podobnou $\lambda$ násobku derivace ztrátové funkce,
        bude produkovat obrázky velice podobné původním, ale s mírným poškozením.
    \end{itemize}

\end{frame}

\end{document}