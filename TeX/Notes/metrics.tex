\documentclass[czech]{article}

\usepackage{amsmath}
\usepackage{amsfonts}
\usepackage{babel}

\usepackage[x11names]{xcolor}
\usepackage{framed}
\usepackage{quoting}

\selectlanguage{czech}

\author{Pavel Jakš}
\title{Vzdálenostní metriky používané pro měření vzdáleností mezi obrázky}

\begin{document}

\maketitle

\section*{Úvod}

Pod pojmem metrika na prostoru $X$ si každý matematik představí zobrazení $\rho : X \times X \rightarrow [0, + \infty)$
splňující
\begin{enumerate}
    \item $\rho(x, y) = 0 \iff x = y \quad \forall x, y \in X$,
    \item $\rho(x, y) = \rho(y, x) \quad \forall x, y \in X$,
    \item $\rho(x, z) \leq \rho(x, y) + \rho(y, z) \quad \forall x, y, z \in X$.
\end{enumerate}

Taková metrika může být na lineárním prostoru $V$ nad číselným tělesem (pro naše účely zůstaňme nad $\mathbb{R}$)
snadno zadána pomocí normy,
která je buď indukována skalárním součinem v případě pre-Hilbertových prostorů,
nebo dána vlastnostmi, že se jedná o zobrazení $\|.\| : V \rightarrow [0, + \infty)$
a splňuje:
\begin{enumerate}
    \item $\|x\| = 0 \iff x = 0 \quad \forall x \in V$,
    \item $\|\alpha x\| = |\alpha| \cdot \|x\| \quad \forall \alpha \in \mathbb{R}, \forall x \in V$,
    \item $\|x + y\| \leq \|x\| + \|y\| \quad \forall x, y \in V$.
\end{enumerate}
S metrikami, které jsou tzv. indukované normami se setkáme.


\section{Metriky indukované $l_p$ normami}

Vzhledem k tomu, že obrázky, které jsou středem naší pozornosti,
lze reprezentovat jako tenzory standardně o rozměrech $C \times W \times H$,
kde $C$ značí počet kanálů (nejčastěji kanály po řadě pro červenou, zelenou a modrou barvu),
$W$ označuje šířku a $H$ výšku, tak lze na tyto tenzory vpustit $L^p$ normy.
Pro $p \in [1, + \infty)$ je $L^p$ norma z $f \in L_p(X, \mu )$
definována vztahem:
\begin{equation*}
    \|f\|_p = \left(\int_X |f|^p \mathrm{d} \mu \right)^{\frac{1}{p}}.
\end{equation*}

Pro naše obrázky lze za $X$ vzít $\{1, ... C\} \times \{1, ..., W\} \times \{1, ..., H\}$ a za $\mu$ \emph{počítací míru}.
Potom naše $L^p$ norma přejde v $l_p$ normu, která má pro naše obrázky, tedy tenzory $x \in \mathbb{R}^{C \times W \times H}$, tvar:
\begin{equation}
    \|x\|_p = \left( \sum_{i=1}^{C} \sum_{j=1}^{W} \sum_{k=1}^{H} |x_{i, j, k}|^p \right)^{\frac{1}{p}}.
\end{equation}

Trochu mimo stojí $l_{\infty}$ norma, která má tvar pro tenzor $x \in \mathbb{R}^{C \times W \times H}$:
\begin{equation}
    \|x\|_\infty = \max_{i \in \{1, ..., C\}} \max_{j \in \{1, ..., W\}} \max_{k \in \{1, ..., H\}} |x_{i, j, k}|.
\end{equation}

A úplně mimo stojí $L_0$ norma, která svou povahou \emph{není} norma ve smyslu výše uvedené definice,
ale pro účely porovnávání obrázků se používá rozdíl obrázků v této pseudo-normě, proto ji zde zmiňuji:
\begin{equation}
    \|x\|_0 = |\{x_{i, j, k} \neq 0\}|.
\end{equation}

\section{MSE a RMSE}

Vzdálenosti, které mají blízko k metrikám indukovaným $l_2$ normou, jsou \emph{MSE} (z anglického \emph{Mean Squared Error})
a \emph{RMSE} (z anglického \emph{Root Mean Squared Error}).
Pro tenzory $x, \tilde{x} \in \mathbb{R}^{C \times W \times H}$ mají definici:
\begin{align}
    \operatorname{MSE}(x, \tilde{x}) &= \frac{1}{C W H} \sum_{i=1}^C \sum_{j=1}^W \sum_{k=1}^H | x_{i, j, k} - \tilde{x}_{i, j, k} |^2 \\
    \operatorname{RMSE}(x, \tilde{x}) &= \left(\frac{1}{C W H} \sum_{i=1}^C \sum_{j=1}^W \sum_{k=1}^H | x_{i, j, k} - \tilde{x}_{i, j, k} |^2 \right)^{\frac{1}{2}}
\end{align}

\section{Wassersteinova vzdálenost}

Buď $(M, d)$ metrický prostor, který je zároveň \emph{Radonův}. Zvolme $p \in [1, + \infty)$.
Potom máme \emph{Wassersteinovu $p$-vzdálenost} mezi dvěma pravděpodobnostními mírami $\mu$ a $\nu$ na $M$,
které mají konečné $p$-té momenty,
jako:
\begin{equation}
    W_p (\mu, \nu) = \left( \inf_{\gamma \in \Gamma(\mu, \nu)} \mathbb{E}_{(x, y) \sim \gamma} \operatorname{d}(x, y)^p \right)^{\frac{1}{p}},
\end{equation}
kde $\Gamma(\mu, \nu)$ je množina všech sdružených pravděpodobnostních měr na $M \times M$,
které mají po řadě $\mu$ a $\nu$ za marginální pravděpodobnostní míry \cite{vaserstejn}.

Jak to souvisí s obrázky?
Přes doprvní problém.

\section{PSNR}

Vzdálenost označená zkratkou \emph{PSNR} z anglického \emph{Peak Signal-to-Noise Ratio}
vyjadřuje vztah mezi obrázkem $x \in \mathbb{R}^{C \times W \times H}$
a jeho pokažením $\tilde{x} \in \mathbb{R}^{C \times W \times H}$ za přidání šumu.
Definice je následující:
\begin{align}
    \operatorname{PSNR}(x, \tilde{x}) &= 10 \cdot \operatorname{log}_{10} \left( \frac{\|x\|_{\infty}^2}{\operatorname{MSE}(x, \tilde{x})} \right), \\
    &= 20 \cdot \operatorname{log}_{10} \left( \frac{\|x\|_{\infty}}{\operatorname{RMSE}(x, \tilde{x})} \right).
\end{align}
Jak je vidět, prohození $x$ a $\tilde{x}$ povede ke změně hodnoty $\operatorname{PSNR}$, tato vzdálenost tedy není metrická.

\section{SSIM}

Pod zkratkou \emph{SSIM} (\emph{Structural Similarity Index Measure})
se rozumí následující vzdálenost:
\begin{equation}
    \operatorname{SSIM}(x, \tilde{x}) = \frac{(2 \mu_x \mu_{\tilde{x} + C_1})(2 \sigma_{x \tilde{x}} + C_2)}{(\mu_x^2 + \mu_{\tilde{x}}^2 + C_1)(\sigma_x^2 + \sigma_{\tilde{x}}^2 + C_2)},
\end{equation}
kde $\mu$ je průměr hodnot pixelů $x$, resp. $\tilde{x}$,
$\sigma_{x \tilde{x}}$ je nestranný odhad kovariance mezi $x$ a $\tilde{x}$,
$\sigma^2$ je nestranný odhad rozptylu $x$, resp. $\tilde{x}$
a $C_1, C_2$ jsou konstanty pro stabilitu dělení volené přímo úměrně dynamickému rozsahu.

Máme-li dva obrázky, tak za $x$ a $\tilde{x}$ do vzorce pro $\operatorname{SSIM}$ se standardně volí jakási okna obrázků.
To znamená, že za celkovou vzdálenost mezi dvěma obrázky volíme průměr přes všechna okna předem zvolené velikosti.


\begin{thebibliography}{1}
	\addcontentsline{toc}{chapter}{Literatura}

\bibitem{vaserstejn} L. Vaserstein,
\emph{Markov processes over denumerable products of spaces, describing large systems of automata}.
Problemy Peredači Informacii 5, 1969.

\end{thebibliography}

\end{document}
