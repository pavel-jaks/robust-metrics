\documentclass[czech]{article}

\usepackage{amsmath}
\usepackage{amsfonts}
\usepackage{babel}

\usepackage[x11names]{xcolor}
\usepackage{framed}
\usepackage{quoting}

\selectlanguage{czech}

\author{Pavel Jakš}
\title{Vzdálenostní metriky používané pro měření vzdáleností mezi obrázky}

\begin{document}

\maketitle

\section*{Úvod}

Pod pojmem metrika na prostoru $X$ si každý matematik představí zobrazení $\rho : X \times X \rightarrow [0, + \infty)$
splňující
\begin{enumerate}
    \item $\rho(x, y) = 0 \iff x = y \quad \forall x, y \in X$,
    \item $\rho(x, y) = \rho(y, x) \quad \forall x, y \in X$,
    \item $\rho(x, z) \leq \rho(x, y) + \rho(y, z) \quad \forall x, y, z \in X$.
\end{enumerate}

Taková metrika může být na lineárním prostoru $V$ nad číselným tělesem (pro naše účely zůstaňme nad $\mathbb{R}$
snadno zadána pomocí normy,
která je buď indukována skalárním součinem v případě pre-Hilbertových prostorů,
nebo dána vlastnostmi, že se jedná o zobrazení $\|.\| : V \rightarrow [0, + \infty)$
a splňuje:
\begin{enumerate}
    \item $\|x\| = 0 \iff x = 0 \quad \forall x \in V$,
    \item $\|\alpha x\| = |\alpha| \cdot \|x\| \quad \forall \alpha \in \mathbb{R}, \forall x \in V$,
    \item $\|x + y\| \leq \|x\| + \|y\| \quad \forall x, y \in V$.
\end{enumerate}
S metrikami, které jsou tzv. indukované normami se setkáme.


\section{Metriky indukované $l_p$ normami}

Vzhledem k tomu, že obrázky, které jsou středem naší pozornosti,
lze reprezentovat jako tenzory standardně o rozměrech $C \times W \times H$,
kde $C$ značí počet kanálů (nejčastěji kanály po řadě pro červenou, zelenou a modrou barvu),
$W$ označuje šířku a $H$ výšku, tak lze na tyto tenzory vpustit $L^p$ normy.
Pro $p \in [1, + \infty)$ je $L^p$ norma z $f \in L_p(X, \mu )$
definována vztahem:
\begin{equation*}
    \|f\|_p = \left(\int_X |f|^p \mathrm{d} \mu \right)^{\frac{1}{p}}.
\end{equation*}

Pro naše obrázky lze za $X$ vzít $\{1, ... C\} \times \{1, ..., W\} \times \{1, ..., H\}$ a za $\mu$ \emph{počítací míru}.
Potom naše $L^p$ norma přejde v $l_p$ normu, která má pro naše obrázky, tedy tenzory $x \in \mathbb{R}^{C \times W \times H}$, tvar:
\begin{equation}
    \|x\|_p = \left( \sum_{i=1}^{C} \sum_{j=1}^{W} \sum_{k=1}^{H} |x_{i, j, k}|^p \right)^{\frac{1}{p}}.
\end{equation}

Trochu mimo stojí $l_{\infty}$ norma, která má tvar pro tenzor $x \in \mathbb{R}^{C \times W \times H}$:
\begin{equation}
    \|x\|_\infty = \max_{i \in \{1, ..., C\}} \max_{j \in \{1, ..., W\}} \max_{k \in \{1, ..., H\}} |x_{i, j, k}|.
\end{equation}

A úplně mimo stojí $L_0$ norma, která svou povahou \emph{není} norma ve smyslu výše uvedené definice,
ale pro účely porovnávání obrázků se používá rozdíl obrázků v této pseudo-normě, proto ji zde zmiňuji:
\begin{equation}
    \|x\|_0 = |\{x_{i, j, k} \neq 0\}|.
\end{equation}

\end{document}
